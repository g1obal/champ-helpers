\documentclass[a4paper,12pt]{report}
\usepackage[utf8]{inputenc}
\usepackage[a4paper, top=2.5cm, left=2.5cm, right=2.5cm, bottom=2.5cm]{geometry}

\usepackage{indentfirst}
\usepackage{hyperref}
\usepackage{amsmath}
\usepackage{graphicx}
\usepackage{verbatim}
\usepackage{color}
\usepackage{subfigure}
\usepackage[toc,page]{appendix}
\usepackage[nottoc]{tocbibind}

% Title Page
\title{Effective Atomic Units}
\author{Gökhan Öztarhan}
\date{17 September 2020}


\begin{document}
\maketitle

\chapter*{Effective Atomic Units}

Calculation of atomic units may vary for several materials. Since many aspects
of atomic units consist of quantities depending on the materials,
those quantities should be taken into account. The most important quantities
are effective mass of the electrons $m^{*}$ and the dielectric constant $\kappa$
for a given material. Additionally, Landé g-factor $g^{*}$ should be known if
there are calculations related to magnetism. In the following sections,
atomic unit calculations are given.

\section*{Atomic unit of length, Bohr radius} \label{bohr}
Bohr radius is given as
\begin{align}
    a_{0} = \frac{4 \pi \varepsilon_{0} \hbar^{2}}{m_{e} e^{2}}
          = 5.291772109 \times 10^{-11} \ \textrm{m}
\end{align}
where $\varepsilon_{0}$ is the vacuum permitivity, $\hbar$ is the reduced
Planck's constant, $m_{e}$ is the mass of the free electrons and $e$ is the
charge of the electrons. For calculation of atomic units, we multiply the
given value of Bohr radius by the ratios of the electric permitivity and the
electron masses instead of calculating whole equation in order to preserve the
numerical accuracy as much as possible. Let us define ratio of the effective
mass of the electrons inside the material relative to the free electron mass as
\begin{align}
    m_{r} = \frac{m^{*}}{m_{e}}
\end{align}
On the other hand, the dielectric constant, or the relative permitivity, is
already the ratio of the permitivity of the material $\varepsilon$ relative to
the vacuum permivitivy $\varepsilon_{0}$, and expressed as
\begin{align}
    \kappa = \frac{\varepsilon}{\varepsilon_{0}}
\end{align}
If the Bohr radius $a_{0}$ is multiplied by $\kappa/m_{r}$, the effective Bohr
radius is obtained as below.
\begin{align}
    a^{*}_{0} &= a_{0} \frac{\kappa}{m_{r}} \nonumber \\
              &= \frac{4 \pi \varepsilon_{0} \hbar^{2}}{m_{e} e^{2}}
              \frac{\varepsilon}{\varepsilon_{0}} \frac{m_{e}}{m^{*}}
              = \frac{4 \pi \varepsilon \hbar^{2}}{m^{*} e^{2}}
\end{align}
Thus, the effective atomic unit of length is 
\begin{align} \label{effective_bohr}
    \boxed{ a^{*}_{0} = a_{0} \frac{\kappa}{m_{r}} }
\end{align} 
The conversion of a length $\ell$ to atomic units for a given material is
the following.
\begin{align}
    \boxed{ \ell^{*} = \frac{\ell}{a^{*}_{0}} }
\end{align} 
Therefore, using equation \ref{effective_bohr};
\begin{align}
    \boxed{ \ell^{*} = \ell \left( a_{0} \frac{\kappa}{m_{r}} \right) ^{-1}}
\end{align} 

\section*{Atomic unit of energy, Hartree energy} \label{hartree}
The atomic unit of energy, the Hartree energy, can be expressed as
\begin{align}
    E_{h} = m_{e} \left( \frac{e^{2}}{4 \pi \varepsilon_{0} \hbar} \right)^{2}
          = 27.211386245988 \ \textrm{eV}
\end{align}
We will calculate the effective Hartree energy with a similar calculation in
previous section. The effective Hartree energy for a given material
can be calculated via multiplying the Hartree energy by $m_{r}/\kappa^{2}$,
where $m_{r} = m^{*}/m_{e}$ and $\kappa = \varepsilon/\varepsilon_{0}$.
\begin{align}
    E^{*}_{h} &= E_{h} \frac{m_{r}}{\kappa^{2}} \nonumber \\
              &= m_{e} \left( \frac{e^{2}}{4 \pi \varepsilon_{0} \hbar} \right)^{2}
              \frac{m^{*}}{m_{e}} \left(\frac{\varepsilon_{0}}{\varepsilon}\right)^{2}
              = m^{*} \left( \frac{e^{2}}{4 \pi \varepsilon \hbar} \right)^{2}
\end{align}
Thus, the effective atomic unit of energy is
\begin{align} \label{effective_hartree}
    \boxed{ E^{*}_{h} = E_{h} \frac{m_{r}}{\kappa^{2}} }
\end{align}
The conversion of an energy $E$ to atomic units for a given material is
the following.
\begin{align}
    \boxed{ E^{*} = \frac{E}{E^{*}_{h}} }
\end{align} 
Therefore, using \ref{effective_hartree};
\begin{align}
    \boxed{ E^{*} = E \left( E_{h} \frac{m_{r}}{\kappa^{2}} \right)^{-1} }
\end{align} 

\section*{Effective atomic units for GaAs}
The effective mass of the electrons in GaAs is $m^{*}=0.067m_{e}$, the
dielectric constant is $\kappa=12.4$, and the Landé g-factor is $g^{*} = -0.44$
as given in \cite{devrimguclu_1}. Thus, the corresponding atomic units are
\begin{align}
a^{*}_{0} &= 9.793727485 \times 10^{-9} \ \textrm{m} \nonumber \\
E^{*}_{h} &= 0.011857198741 \ \textrm{eV} \nonumber
\end{align}

\bibliographystyle{plain}
\bibliography{effective_atomic_units}


\end{document}          

